\thusetup{
  %******************************
  % 注意:
  %   1. 配置里面不要出现空行
  %   2. 不需要的配置信息可以删除
  %******************************
  %
  %=====
  % 秘级
  %=====
  secretlevel={秘密},
  secretyear={10},
  %
  %=========
  % 中文信息
  %=========
  ctitle={ASPCA 一种基于PCA异常空间稀疏化的异常诊断方法},
  cdegree={工学硕士},
  cdepartment={计算机科学与技术系},
  cmajor={计算机科学与技术},
  cauthor={宾行言},
  csupervisor={赵颖副教授},
  % 日期自动使用当前时间,若需指定按如下方式修改:
  % cdate={超新星纪元},
  %
  % 博士后专有部分
  %
  %=========
  % 英文信息
  %=========
  etitle={Abnormal Subspace Sparse PCA for Anomaly Detection and Interpretation},
  % 这块比较复杂,需要分情况讨论:
  % 1. 学术型硕士
  %    edegree:必须为Master of Arts或Master of Science(注意大小写)
  %             “哲学、文学、历史学、法学、教育学、艺术学门类,公共管理学科
  %              填写Master of Arts,其它填写Master of Science”
  %    emajor:“获得一级学科授权的学科填写一级学科名称,其它填写二级学科名称”
  % 2. 专业型硕士
  %    edegree:“填写专业学位英文名称全称”
  %    emajor:“工程硕士填写工程领域,其它专业学位不填写此项”
  % 3. 学术型博士
  %    edegree:Doctor of Philosophy(注意大小写)
  %    emajor:“获得一级学科授权的学科填写一级学科名称,其它填写二级学科名称”
  % 4. 专业型博士
  %    edegree:“填写专业学位英文名称全称”
  %    emajor:不填写此项
  edegree={Master of Science},
  emajor={Computer Science and Technology},
  eauthor={Bin Xingyan},
  esupervisor={Associate Professor Zhao Ying},
  % 日期自动生成,若需指定按如下方式修改:
  % edate={December, 2005}
  %
  % 关键词用“英文逗号”分割
  ckeywords={机器学习, 异常检测, 异常诊断, PCA, 稀疏},
  ekeywords={machine learning, anomaly detection, anomaly analysis, PCA, sparsity}
}

% 定义中英文摘要和关键字
\begin{cabstract}
  主成分分析(PCA)在被运用到异常检测中时,其主要缺点在于其可解释性。在这篇文章中,所介绍的ASPCA模型是一个在可解释性方面有了很大改善的基于PCA的改进模型。模型核心机理是通过构建具有系数稀疏与正交特点的主成分来描述异常空间。
  
  通过在一个模拟数据集以及两个真实数据集上所进行的实验表面,该模型在异常检测精度方面,效果与原始的PCA方法近似,并且可以针对每一个检测出来的异常进行单独分析。

\end{cabstract}

% 如果习惯关键字跟在摘要文字后面,可以用直接命令来设置,如下:
% \ckeywords{\TeX, \LaTeX, CJK, 模板, 论文}

\begin{eabstract}
	The main shortage of principle component analysis (PCA) based anomaly detection models is their interpretability. In this paper, our goal is to propose an interpretable PCA- based model for anomaly detection and interpretation. The propose ASPCA model constructs principal components with sparse and orthogonal loading vectors to represent the abnormal subspace, and uses them to interpret detected anomalies. Our experiments on a synthetic dataset and two real world datasets showed that the proposed ASPCA models achieved comparable detection accuracies as the PCA model, and can provide interpretations for individual anomalies.
  
\end{eabstract}

% \ekeywords{\TeX, \LaTeX, CJK, template, thesis}
