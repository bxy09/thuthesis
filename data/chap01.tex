\chapter{绪论}
\label{cha:intro}
PCA是一种常用的基于统计的异常检测模型。很多研究者尝试将该方法运用于网络流量分析与攻击检测\cite{Anomaly-Survey-09,XuWeiGoogle,Lakhina-2005-sigcomm,jiang2013family},但该方法不仅仅只是学术领域的尝试,在生产系统中,例如美国最大的版权电视节目网站Netflix就在使用PCA算法作为他们的安全检测算法\cite{netflix-outlier}。除了计算机领域,PCA模型还可以被运用在金融领域\cite{jiang2013family}以及航天领域\cite{dutta2007distributed}。在微软机器学习云服务中,对于异常检测它推荐了两种算法,对于特征数量比较有限的场景,微软的推荐就是基于PCA的异常检测方法\cite{microsoft-cheatsheet}。

在实际应用中,如果在异常检测时,能够阐释其判断的依据,则能够帮助解决这些异常。例如航天系统的某个异常,如果可以描述出具体的判断依据,则可以更快的进行问题的排查,更快的恢复系统工作。例如对于网络入侵的检测,如果能给出判断依据,则可以由系统管理员对系统的相应环节进行更好的安全加固。我们把这种可以对于每一个检测出的异常都能够给出人所能解读其依据的能力,称为{\bf 异常解释(anomaly interpretation)}。一个具有异常解释能力的异常检测系统将会更具有实用价值。

然而,传统的基于PCA模型的异常检测模型不能很好的进行异常解释\cite{XuWei-SOSP,PCA-Sensitivity}。

PCA分辨异常依据的是某个数据点不符合数据集主流的统计规律。而PCA所挖掘的主流统计规律是在原始数据空间中,存在一个低维的子空间能够容纳数据的主要波动。当数据点不能完全被这样的一个低维子空间所容纳时,模型判断它为异常。由于最终指示数据点是否为异常的指标与原始数据之间存在很多步骤,并且由于对低维子空间的描述非常复杂,导致中间处理步骤的计算也是非常复杂的,因此这样的计算过程很难进行解释。利用PCA进行异常检测的方法被人视作是一种黑箱方法\cite{XuWei-SOSP}。

人们为了解决这个问题,从不同的方向进行了尝试。\cite{XuWei-SOSP} 在通过用PCA进行异常检测后,增加了一个新的步骤,利用决策树模型来解释检测出的异常,决策树模型具有很好的可解释性。但是,这种间接的解释方式,不能真正解释PCA异常检测方法判断异常的原因\cite{PCA-Sensitivity}。另一个工作\cite{jiang2013family}提出了联合稀疏化PCA(JSPCA)的新模型,该模型提出用一个新的子空间去近似描述数据空间中不能被代表主流规律的子空间所接受的部分,这个新空间将只涉及到少数原本空间的特征维度。这种方法定位了哪些特征维度会与检测数据所有的异常点相关。但对于每个单独的异常记录,尤其是完全不同的异常记录,该模型却只能给出相同的模糊指示。总之,现有方法无法准确地,直接地给出每个异常记录被判断为异常的依据,无法进行异常解释。

我们的目的是设计一种基于PCA的改进模型,使得它能够同时进行异常检测与异常解释。模型同样是尝试获取一个新的子空间来描述数据中不能被主流规律子空间所接受的部分。这样的子空间具有以下特点:与主流规律子空间正交,描述该子空间所用的基向量具有正交和稀疏的性质。我们将这种可解释的异常检测模型称为{\bf ASPCA}(Anomaly subspace Sparse PCA)即异常子空间稀疏化PCA。

本文主要工作有,首先,提出了ASPCA模型的两种计算方式。第一种是按照PCA求解一般惯例地,从最主要的主成分开始进行提取,然后再提取剩余的主成分,即先提取体现主流变化的子空间,它所剩余的子空间构成了我们所要的分析异常的子空间;第二种方式与之相反,将从变化量(Variance)最小的主成分开始提取,即直接获取我们用来分析异常的子空间。其中,第二种方式将会在异常子空间下获取更好的稀疏性质。本文提出的ASPCA模型是第一个基于PCA模型,且能进行单个异常记录的检测解释能力的模型。其次,我们通过半正定规划(SDP,Semidefinite programming)对提出的模型进行一定程度的放松之后进行优化求解,并且在逐个求出异常子空间的主成分后,还对该子空间以稀疏性为目标进行了全局优化,尝试获得更好的稀疏性质。

本文中的实验通过一个人工数据以及两个真实数据展示了ASPCA模型所能取得的与PCA在异常检测方面相似的表现,以及对逐个检测出的异常记录进行解释的能力。

本文之后的章节如下:第二章介绍相关工作,第三章介绍我们所提出的ASPCA模型,第四章介绍优化算法,第五章介绍实验,最后在第六章进行一些总结。


