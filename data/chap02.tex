\chapter{相关工作}
PCA方法最为人所知的用法,是用于特征压缩\cite{jolliffe2002principal},通过PCA对数据的挖掘,可以得到更少的新的特征来描述整个数据集的变化。而同时,也有很多工作将其作为一种异常检测方法\cite{dunia1997multi,Anomaly-Survey-09}。Wei Xu 等人将该技术用于分析日志,并且将其运用在一个游戏服务器上,还利用该方法检测出了Hadoop File System(HDFS)的一些问题\cite{XuWei-SOSP}。该工作还被运用在了Google的生产系统中\cite{XuWeiGoogle}。Ryohei Fujimaki 等人将 kernel PCA 用在了宇宙飞船的异常检测问题上 \cite{kernelPCA-Space}。
还有很多研究者将PCA技术用在了网络入侵检测上\cite{Lakhina-2005-sigcomm,Lakhina-2004-sigcomm,jiang2013family,PCA-KDD99-2006}。Shyu, M.L. 等人利用大成分(数据变化大的部分)以及小成分(数据变化小的部分,也就是本工作所定义的异常子空间)形成 Mahalanobis距离,取代一般所用的小成分空间的欧氏距离,并且采用鲁棒PCA来提升无监督异常检测的表现\cite{shyu2003novel}。Anukool Lakhina 等人将PCA模型运用在了网络洪流检测问题当中,所分析的数据是关于每一个源头-目的地二元组与时间的矩阵\cite{Lakhina-2005-sigcomm,Lakhina-2004-sigcomm}。首先,他们主要是分析通信的流量大小\cite{Lakhina-2004-sigcomm},之后他们将通信流量的熵也作为描述二元组的另一个特征引入进来,形成多子空间的PCA检测\cite{Lakhina-2005-sigcomm}。Ling Huang 等人还尝试设计了一种在线PCA监测模型,模型考虑了可扩展性以及分布式布置下的通信效率\cite{huang2006network,INFOCOM-Distributed-PCA}。

当将PCA作为一种特征压缩工具时,其主要缺点在于糟糕的可解释性。Ian Jolliffe 等人引入了稀疏PCA的概念,在这种概念下,对提取出的特征参数向量进行了稀疏性方面的限制\cite{SPCA-2003}。自此,产生了很多求解稀疏PCA问题的算法,例如 \cite{SPCA-2006} 和 \cite{SPCA-SDP}。Hui zou 等人将稀疏PCA问题转化为了一个带elastic net正规项的回归问题,该问题可以通过交替最小化(alternating minimization)的算法进行求解\cite{SPCA-2006}。Alexandre d'Aspremont 等人通过对原问题的一些条件的放松,提出了通过半正定规划(SDP)求解稀疏PCA问题,该方法从大到小逐个求解主成分\cite{SPCA-SDP}。

当将PCA作为一种异常检测工具时,糟糕的可解释性同样是它的缺点 \cite{PCA-Sensitivity,XuWei-SOSP}。Ruoyi Jiang 等人受到稀疏PCA启发,引入了连接稀疏PCA的方法来解决异常定位问题\cite{jiang2011JSPCA,jiang2013family}。他们的工作采用的是交替最小化的框架\cite{SPCA-2006}来解决他们所提出的优化问题。 Wei Xu 等人也尝试了解决PCA异常检测可解释性查的问题。他们讲检测模型所得到的异常记录标记上标签,将其与正常记录作为训练数据输入到决策树模型。决策树模型是一个具有很好的可解释性的模型,但正如我们之后的实验结果所表现的那样,这种方案所找到的解释可能与PCA模型相悖,因为他体现的并不是PCA模型本身的依据,而是PCA检测结果的统计表现,也就是一种“事后诸葛亮”的分析,并不能反映检测本身的思路。
